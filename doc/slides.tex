\documentclass[paper=screen,orient=landscape,style=simple]{powerdot}
\usepackage{listings}
\usepackage{amssymb}
\usepackage{times}

\lstset{
% language=C++,
 basicstyle=\footnotesize,
 morekeywords={array},
 escapeinside=`'
}

\title{Beyond PLY}
\author{Daniel Pfeifer}
\date{VMML@IFI Group Meeting on \today}


\begin{document}

\maketitle

\section{Random Access PLY}

\begin{slide}[method=file]{PLY 1.0}
\linespread{0}
\begin{lstlisting}
  ply
  format ascii 1.0
  comment made by anonymous
  comment this file is a cube
  element vertex 8
  property float32 x
  property float32 y
  property float32 z
  element face 6
  property list uint8 int32 vertex_index
  end_header
  0 0 0
  0 0 1
  0 1 1
  0 1 0
  1 0 0
  1 0 1
  1 1 1
  1 1 0
  4 0 1 2 3
  4 7 6 5 4
  4 0 4 5 1
  4 1 5 6 2
  4 2 6 7 3
  4 3 7 4 0
\end{lstlisting}
\end{slide}

\begin{slide}{PLY 1.0}
  \begin{itemize}
    \item format := ascii $|$ binary
    \item element := name size property+
    \item property := name ( scalar $|$ list )
    \item list := scalar scalar
  \end{itemize}

  \textbf{Problem:}\\
  Variable sized content limits parsing to \textbf{forward} only!\\
  There is no way to implement \textbf{random access}.\\
  (Analogous to \textbf{Forward} resp. \textbf{RandomAccess} iterator concepts).
\end{slide}

\begin{slide}{PLY 2.0}
  \begin{itemize}
    \item format is always binary, native byteorder
    \item dropped support for lists
    \item new support for \textbf{arrays}
    \item array := size scalar
  \end{itemize}

  \textbf{List:}\\
  property array vertex\_indices uint8 uint32\\[2em]

  \textbf{Array:}\\
  property array vertex\_indices 3 uint32
\end{slide}

\begin{slide}[method=file]{PLY 2.0}
\linespread{0}
\begin{lstlisting}
  ply
  format ascii 2.0
  comment made by anonymous
  comment this file is a cube
  element vertex 8
  property float32 x
  property float32 y
  property float32 z
  element face 6
  property array 4 int32 vertex_index
  end_header
  0 0 0
  0 0 1
  0 1 1
  0 1 0
  1 0 0
  1 0 1
  1 1 1
  1 1 0
  0 1 2 3
  7 6 5 4
  0 4 5 1
  1 5 6 2
  2 6 7 3
  3 7 4 0
\end{lstlisting}
\end{slide}

\section{Why PLY?}

\begin{slide}{Why PLY?}
Quoting the original PLY doc:\\[1em]
``Our goal is to provide a format that is \textbf{simple and easy} to implement
but that is general enough to be useful for a wide range of models.''

\begin{itemize}
  \item Was that goal achieved?
  \item Is it easy to implement?
  \item Is it simple to implement?
\end{itemize}
\end{slide}

\begin{slide}{easy and simple?}
All the identifiers are at the beginning of a line.\\
It is possible to parse it like this:

\begin{itemize}
  \item while(not EOF)
    \begin{itemize}
      \item read line
      \item tokenize line
      \item if token == "format"
        \begin{itemize}
          \item ...
        \end{itemize}
      \item else if token == "comment"
        \begin{itemize}
          \item ...
        \end{itemize}
      \item ...
    \end{itemize}
\end{itemize}
\end{slide}

\begin{slide}{easy and simple?}
\begin{itemize}
  \item only minimal programming skills required\\
        $\rightarrow$ easy: YES
  \item Lots of code required:\\
        $\rightarrow$ simple: NO!!
\end{itemize}
\end{slide}

\begin{slide}[method=file]{simplify}
Generate parser from formal grammar:

\begin{lstlisting}
ply      ::= "ply" EOL "format" format DOUBLE EOL element* "end_header" EOL
element  ::= "element" STRING INT EOL property*
property ::= "property" (list | scalar) STRING EOL
list     ::= "list" size scalar
format   ::= "ascii"|"binary_little_endian"|"binary_big_endian"
size     ::= "uint8"|"uint16"|"uint32"|"uint64"
scalar   ::= size|"int8"|"int16"|"int32"|"int64"|"float32"|"float64"
\end{lstlisting}
\end{slide}

\begin{slide}[method=file]{Qi}
C++ implementation with Qi (Boost.Spirit):

\begin{lstlisting}
start %= qi::eps > "ply" > qi::eol
  > "format" > format_ > qi::double_ > qi::eol
  > *element_
  > "end_header" > qi::eol;

element_ %= "element"
  > *(ascii::char_ - qi::int_)
  > qi::int_ > qi::eol
  > *property_;

property_ %= "property" > (list_ | scalar_)
  > *(ascii::char_ - qi::eol) > qi::eol;

list_ %= "list" > size_ > scalar_;
\end{lstlisting}
\end{slide}

\begin{slide}{easy and simple?}
\begin{itemize}
  \item knowledge about formal languages required (2nd semester)\\
        $\rightarrow$ easy: YES
  \item seven rules for the complete PLY grammar:\\
        $\rightarrow$ simple: YES
\end{itemize}

But this would hold true for any grammar!

So let's change it...
\end{slide}

\begin{slide}[method=file]{new grammar}
\linespread{0}
\begin{lstlisting}
start
  %= qi::eps
  > "#define" > endian
  > *element_
  > qi::eoi
  ;
element_
  %= "typedef struct {"
  > *attribute_
  > '}' > string_ > ',' > string_ > size_ > ';'
  ;
attribute_
  %= scalar_ > string_ > size_ > ';'
  ;
string_
  %= qi::lexeme[+(ascii::alnum | qi::char_('_'))]
  ;
size_
  %= ('[' > qi::uint_ > ']') | qi::eps(qi::_val = 1)
  ;
\end{lstlisting}
\end{slide}

\begin{slide}{easy and simple?}
\begin{itemize}
  \item knowledge about formal languages required (2nd semester)\\
        $\rightarrow$ easy: YES
  \item seven rules for the complete PLY grammar:\\
        $\rightarrow$ simple: YES
\end{itemize}

But it you will get some bonus...
\end{slide}

\begin{slide}[method=file]{Bonus}
\begin{lstlisting}
#define LITTLE_ENDIAN
 
typedef struct {
   float32 x;
   float32 y;
   float32 z;
} vertex, vertices[8];
 
typedef struct {
   uint32 indices[3];
} face, faces[6];
\end{lstlisting}

The file header is now valid C code!
\end{slide}

\end{document}

